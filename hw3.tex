\documentclass{article}
\usepackage[russian]{babel}
\usepackage{amsfonts}
\usepackage{graphicx}
\usepackage{amsfonts}
\usepackage{listings}
\usepackage{mathtools}
\usepackage[T1]{fontenc}
\usepackage[export]{adjustbox}
\graphicspath{{pictures/}}
\DeclareGraphicsExtensions{.pdf,.png,.jpg}
\usepackage{amsthm}
\usepackage[14pt]{extsizes}
\vspace{50mm}
\setlength{\parskip}{1.2em}
\linespread{1.2}
\usepackage[left=25mm, top=25mm, right=25mm, bottom=25mm, nohead]{geometry}
\newtheorem{theorem}{Утверждение}

\begin{document}
\thispagestyle{empty}
\begin{center}
\large \textbf{МИНОБРНАУКИ РОССИИ}\\
\normalsize{федеральное государственное бюджетное образовательное\\
учреждение высшего образования}\\
\large \textbf{Национальный исследовательский университет «МЭИ»}\\
\noindent\rule{15cm}{0.4pt}
\end{center}
\begin{flushright}
\large \textbf {Институт \underline {ИВТИ}}\\
\large \textbf {Кафедра \underline {ПМИИ}}\\
\end{flushright}
\begin{center}
\large \textbf{Теоритические модели вычислений}\\
\large{ДЗ №3: Машины Тьюринга и квантовые вычисления}\\
\end{center}
\hfill \break
\begin{flushright}
{Выполнил: студент группы А-13а-19 
\\Тулинов А.В.\\
Преподаватель: Ивлиев С.А.}\\
\end{flushright}
\hfill \break
\hfill \break
\hfill \break
\begin{center}
Москва, 2022 г.
\end{center}

\newpage
\textbf{\Large 2 Машины Тьюринга}\\\\
\hfill \break
\textbf{\Large 2.1 Операции с числами}\\\\
Реализуйте машины Тьюринга, которые позволяют выполнять следующие операции: \\
\hfill \break
\textbf{1.} Сложение двух унарных чисел (1 балл) \\
\hfill \break
\normalsize \underline{Решение:}\\
\hfill \break
\includegraphics[scale = 1.0] {mt2.1_1}\\
\hfill \break
\newpage
\textbf{Листинг:}\\
\hfill \break
\includegraphics[scale = 1.0] {1_yaml}\\
\hfill \break
Описание \textbf{.yaml} к соответствующей задаче находится в папке MT.\\\\
\hfill \break
\newpage
\textbf{2.} Умножение унарных чисел (1 балл) \\
\hfill \break
\normalsize \underline{Решение:}\\\\
\hfill \break
\includegraphics[scale = 0.8] {mt2.1_2}\\
\hfill \break
\newpage
\textbf{Листинг:}\\
\hfill \break
\includegraphics[scale = 0.9] {2_yaml}\\
\hfill \break
Описание \textbf{.yaml} к соответствующей задаче находится в папке MT.\\\\

\newpage
\textbf{\Large 2.2 Операции с языками и символами}\\\\
Реализуйте машины Тьюринга, которые позволяют выполнять следующие операции: \\
\hfill \break
\textbf{1. }Принадлежность  к языку L = \(\{0^n1^n2^n\}, n \geq 0\) (0.5 балла)\\
\hfill \break
\normalsize \underline{Решение:}\\
\hfill \break
\includegraphics[scale = 0.7] {mt2.2_1}\\
\newpage
\textbf{Листинг:}\\
\hfill \break
\includegraphics[scale = 0.9] {3_yaml}\\
\hfill \break
Описание \textbf{.yaml} к соответствующей задаче находится в папке MT.\\\\
\textbf{2. }Проверка соблюдения правильности скобок в строке (минимум 3 вида скобок) (0.5 балла)\\
\hfill \break
\normalsize \underline{Решение:}\\
\hfill \break
\includegraphics[scale = 1.0] {mt2.2_2}\\
\newpage
\textbf{Листинг:}\\
\hfill \break
\includegraphics[scale = 1.0] {4_yaml}\\
\hfill \break
Описание \textbf{.yaml} к соответствующей задаче находится в папке MT.\\\\
\newpage
\textbf{3. }Поиск минимального по длинне слова в строке (слова состоят из символов 1 и 0 и разделены пробелом) (1 балл)\\
\hfill \break
\normalsize \underline{Решение:}\\
\hfill \break
\includegraphics[scale = 1.2] {mt2.2_3}\\
\newpage
\textbf{Листинг:}\\
\hfill \break
\includegraphics[scale = 1.0] {5_yaml}\\
\hfill \break
Описание \textbf{.yaml} к соответствующей задаче находится в папке MT.\\
\newpage

\textbf{\Large 3 Квантовые вычисления}\\\\
\hfill \break
\textbf{\Large 3.1 Генерация суперпозиций (1 балл)}\\\\
\hfill \break
Дано \(N\) кубитов \((1 \leq N \leq 8)\) в нулевом состоянии \(|0...0\rangle\). Также дана некоторая последовательность битов, которая задаёт ненулевое базисное состояние размера \(N\). Задача получить суперпозицию нулевого состояния и заданного.\\\\
\(|S\rangle\) = \(\frac{1}{\sqrt{2}}\) \((|0 ... 0\rangle + |\psi\ \rangle)\)\\\\
То есть требуется реализовать операцию, которая принимает на вход:\\
1. Массив кубитов \(q_s\)\\
2. Массив битов \(bits \) описывающих некоторое состояние \(|\psi\ \rangle\). Этот массив имеет тот же самый размер, что и \(q_s\).\\ 
Первый элемент этого массива равен \(1\).\\\\
\textbf{Листинг:}\\
\begin{lstlisting}
namespace Solution {
    open Microsoft.Quantum.Primitive;
    open Microsoft.Quantum.Canon;
    operation Solve(qs: Qubit[], bits: Bool[]) : () 
    {
        body 
        { 
            H(qs[0]);
            for (i in 1..Length(qs) - 1) 
			{
                if (bits[i]) 
				{
                    CNOT(qs[0], qs[i]); 
                } 
            }                  
        }
    }
}
\end{lstlisting}

\textbf{\Large 3.2 Различение состояний 1 (1 балл)}\\\\
\hfill \break
Дано \(N\) кубитов \((1 \leq N \leq 8)\), которые могут быть в одном из двух состояний:\\\\
\(|GHZ\rangle\) = \(\frac{1}{\sqrt{2}}\) \((|0 ... 0\rangle + |1 ... 1\rangle)\)\\\\
\(|W\rangle\) = \(\frac{1}{\sqrt{N}}\) \((|10 ... 00\rangle + |01 ... 00\rangle + ... + |00 ... 01\rangle)\)\\\\
Требуется выполнить необходимые преобразования, чтобы точно различить эти два состояния. Возвращать 0, если первое состояние и 1, если второе.\\\\
\textbf{Листинг:}\\
\begin{lstlisting}
namespace Solution {
    open Microsoft.Quantum.Primitive;
    open Microsoft.Quantum.Canon;
    operation Solve(qs: Qubit[]) : Int 
	{
        body 
		{   
            mutable one = 0;
            for (q in qs) 
			{
                if (M(q) == One) 
				{ 
                    set one = one + 1; 
                }
            }
            if (one == 1) 
			{
                return 1;
            } 
			else 
			{
                return 0;
            }                
        }
    }
}
\end{lstlisting}

\textbf{\Large 3.3 Различение состояний 2 (2 балла)}\\\\
\hfill \break
Дано 2 кубита, которые могут быть в одном из двух состояний:\\\\
\(|S_0\rangle\) = \(\frac{1}{2}\) \((|00\rangle + |01\rangle + |10\rangle + |11\rangle)\)\\\\
\(|S_1\rangle\) = \(\frac{1}{2}\) \((|00\rangle - |01\rangle + |10\rangle - |11\rangle)\)\\\\
\(|S_2\rangle\) = \(\frac{1}{2}\) \((|00\rangle + |01\rangle - |10\rangle - |11\rangle)\)\\\\
\(|S_3\rangle\) = \(\frac{1}{2}\) \((|00\rangle - |01\rangle - |10\rangle + |11\rangle)\)\\\\
Требуется выполнить необходимые преобразования, чтобы точно различить эти четыре состояния. Возвращать требуется индекс состояния (от 0 до 3).\\\\
\textbf{Лпстинг:}\\
\begin{lstlisting}
namespace Solution {
    open Microsoft.Quantum.Canon;
    open Microsoft.Quantum.Primitive;
    operation Solve (qs: Qubit[]) : Int
    {
        body 
        {
			H(qs[0]);
			H(qs[1]);
			if (M(qs[0]) == Zero) 
			{
				if (M(qs[1]) == Zero)
				{
					return 0;
				}
				else
				{
					return 1;
				}
			}
			else
			{
				if (M(qs[1]) == Zero)
				{
					return 2;
				}
				else
				{
					return 3;
				}
			}
        }
    }
}
\end{lstlisting}
\end{document}
